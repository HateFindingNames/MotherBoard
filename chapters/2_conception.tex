% LTeX: language=de-DE
\chapter{Konzeption}\label{sec:conception}
	%
	Um Konsistenz mit Literatur und Marktrecherchen sicherzustellen werden im Rahmen dieser Arbeit technische Begriffe aus der Skaterszene genutzt.
	Während Skateboarding einerseits keinesfalls als neuartiges Phänomen zu bezeichnen ist und andererseits in den vergangenen fünf Jahren eine Renaissance erlebt hat, kann, gerade im nicht-englischsprachigen Raum, nicht davon ausgegangen werden, dass alle Lesenden mit der Terminologie vertraut sind.
	Daher wird im Folgenden zunächst Fokus auf eine Begriffskonvention gelegt und in diesem Rahmen funktionale Kernkomponenten eines Skate- bzw. Longboards\footnote{\hspace{1mm} Baulich zwar leicht zu unterscheiden, jedoch aus den gleichen Kernkomponenten und -funktionalitäten aufgebaut.}~erläutert.\par\medskip
	%
	Das Antriebssystem soll in ein vorgegebenes, übergeordnetes System bestehend aus mechanischen und elektronischen Komponenten eingebettet werden.
	Sowohl das übergeordnete System, als auch die Einsatzumgebung definieren konstruktive Einschränkungen, die in einem weiteren Unterkapitel herausgearbeitet werden sollen.\par\medskip
	%
	Zuletzt werden vor dem Hintergrund zuvor festgelegter Rahmenbedingungen und unterstützt durch Markt- und Literaturrecherche Designziele definiert und erste Designideen konkretisiert.
	%
	\section{Funktionale Komponenten eines Longboards}
		%
		Historisch ergaben sich esoterische lautende Namenskonventionen für Teilkomponenten von Skate- und Longboards.
		Während sie im einfachsten Fall unabhängig des Sprachraumes mit ihren jeweiligen englischen Begriffen zu finden sind, weichen Bezeichnungen bisweilen stark von in der Industrie verbreiteten Bezeichnungen ab.
		Um jener Namenskonvention zu folgen, sollen hier zunächst die Teilkomponenten kurz beschrieben werden.\par\medskip
		%
		\begin{figure}[h]
			\centering
			\includesvg[width=.9\textwidth]{Footage/AwesomeBoard Transmission CAD/Drawings/Longboard.svg}
			\caption[Grundlegender Aufbau eines Longboards]{Grundlegender Aufbau eines Longboards. 15~Deck, 16~Truck Bolts, 17~Truck Bolt Nuts, 18~Riser Pad.}\label{fig:longboard}
		\end{figure}
		%
		\Cref{fig:longboard} zeigt den grundlegenden Aufbau eines Longboards.
		Erkennbar sind hier vordergründig das Deck~(15), welches die fahrende Person trägt und hierbei einen Großteil der wirkenden Kräfte aufnehmen muss.
		Je nach Fahrstil werden weichere oder härtere Materialien gewünscht um etwa Unebenheiten des Untergrundes auszugleichen oder die Ausführung von Tricks\footnote{\hspace{1mm} Über die rein laterale Fortbewegung hinausgehende, meist kunstvoll ausgeführte Bewegungen des Boards mit und unter den Füßen.}~positiv zu beeinflussen.

		Meist kommt hier Schichtholz mit oder ohne eingearbeiteten Glas-, Aramid- oder Kohlefasergewebes zum Einsatz, es sind bisweilen aber auch exotischere Materialien wie Aluminium oder ABS\footnote{\hspace{1mm} Acrylnitrilbutadienstyrol.}~vertreten.\nomenclature[A]{ABS}{Acrylnitrilbutadienstyrol}
		In \cref{fig:longboard} befinden sich links und rechts, zentral entlang der Längsachse des Decks angeordnet die Trucks genannten Baugruppen zusammen mit jeweils zwei Rollen.
		Eine mechanisch belastbare Verbindung zum Deck wird durch Truck Bolts~(16) und Truck Bolt Nuts~(17) hergestellt.
		Gegenüber den mit deutlich kleineren Rollen ausgestatteten Skateboards ist es im Longboarding üblich zwischen Deck und Truck Riser Pads~(18) einzusetzen.
		Hierbei handelt es sich um aus flexiblem Material verschiedener Härtegrade gefertigte Pufferplatten mit Doppelfunktion:
		Einerseits unterstützen sie die Entkopplung der Füße von Vibration und verhindern andererseits sogenannte Wheel Bites -~ein Kontakt des Decks mit den Rollen während eines Lenkmanövers mit meist fataler Konsequenz.

		%
		\begin{figure}[h]
			\centering
			\includesvg[width=.9\textwidth]{Footage/AwesomeBoard Transmission CAD/Drawings/Drivetrain - Truck}
			\caption[Explosionsansicht eines der Trucks]{Aufbau eines der Trucks als Explosionsansicht exemplarisch an einem \textsc{Caliper II}. Sichtbar sind hier: 1~Baseplate, 2~Pivot Cup, 3~Kingpin, 4~Hanger, 5~RS Bushing, 6~BS Bushing, 7~BS Washer, 8~RS Washer, 9~Kingpin Nut.}\label{fig:caliper exploded}
		\end{figure}
		%
		Der Aufbau der Trucks selbst wird in \cref{fig:caliper exploded} exemplarisch am Typ \textsc{Caliber II} gezeigt.
		Der Hanger~(4) bildet hier das zentrale Bauteil und muss den Großteil der wirkenden Kräfte aufnehmen.
		Er wird drehbar und gleitend in der Baseplate~(1) gelagert.
		Direkter Kontakt zwischen Hanger und Baseplate hätte erhöhten Abrieb und reduziertes Gleitverhalten zur Konsequenz wozu hier ein meist aus POM\footnote{\hspace{1mm} Polyoxymethylen.}~gefertigter Pivot Cup~(2) eingesetzt wird.
		In Position gehalten wird der Hanger mittels Kingpin~(3) und Kingpin Nut~(9).
		Das Rückstellmoment nach Ende eines Lenkmanövers wird durch zwei Bushings --~vergleichsweise dicke Gummiringe -- erzeugt, die mit der Kingpin-Achse koaxial beidseitig mit dem Hanger in mechanischem Kontakt stehen.
		Deckseitig befindet sich der BS Bushing~(6), straßenseitig angeordnet der RS Bushing.\nomenclature[A]{BS}{Board Side}\nomenclature[A]{RS}{Road Side}
		Die Präfixe stehen für Board Side bzw. Road Side und spiegeln ihre Positionen wider.
		Unterschieden wird hier, da sich verschiedene Geometrien und Härtegrade der Bushings abhängig ihrer Position merklich auf das Fahrgefühl und Lenkvermögen auswirken können\footnote{\hspace{1mm} Durch Drehen der Kingpin Nut und damit einer Änderung der Vorspannung der Bushings~kann hier auch im Feld relativ unkompliziert nachjustiert werden}.
		Um die Lasten flächig auf die Oberfläche der Bushings zu verteilen, werden RS und BS Washer~(7,8), üblicherweise tellerförmige Scheiben, auf der dem Hanger abgewandten Seite platziert.\par\medskip
		%
		\begin{figure}[h]
			\centering
			\includesvg[width=.9\textwidth]{Footage/AwesomeBoard Transmission CAD/Drawings/Drivetrain - Wheels NT}
			\caption[Montage der Rollen an der Achse des Hanger]{Montage der Rollen an der Achse des Hangers. Links in explodierter und rechts in Schnittansicht. Zu sehen sind: 10~Rolle, 11~Kugellager, 12~Speedring, 13~Spacer, 14~Achsmutter.}\label{fig:wheel NT exploded}
		\end{figure}
		%
		Je nach gewünschter Laufruhe und in Abwägung zwischen Traktion und Rollwiderstand werden Rollen verschiedener Größen und Materialien drehbar auf den Hanger-Achsen gelagert.
		Mit Blick auf \cref{fig:wheel NT exploded} wird konzentrisch in den Kern der Rolle~(10) --~vergleichbar mit einer Felge, die das weichere Mantelmaterial trägt --~beidseitig jeweils ein Radialrillenkugellager~(11) platziert.
		Hier hat sich historisch über quasi-sämtliche Rollenbauerten hinweg das 608-Kugellager durchgesetzt.
		Um potenziell den Lagern gegenüber destruktive axiale Kräfte in Kurven oder durch das Anzugsmoment der Achsmutter~(14) zu minimieren, wird, ebenfalls in den Kern der Rolle und zwischen die beiden Lager, der Spacer~(13) angeordnet.
		Er besteht aus hartem Metall, hat einen Innendurchmesser gleich des nominellen Durchmessers der Hanger-Achse, eine Wandstärke von \(\sim \qty{1}{\milli\metre}\) und eine Länge, die gerade so gewählt ist, dass er an beiden Flanken mit den Innenringen der Lager in Kontakt steht, wenn sie beide vollständig im Kern eingelassen sind.
		Ein schleiffreies Laufen der Lager wird durch Speedrings~(12) sichergestellt.
		Ihre Dimensionen entsprechen denen des Spacers, allerdings mit einer deutlich geringeren Länge von nur \qty{1}{\milli\metre}.
		Letztlich werden alle Komponenten mit einer Achsmutter auf der Hanger-Achse fixiert.
		%
	\section{Konstruktive Rahmenbedingungen}\label{sec:constructive limitations}
		%
		Das Antriebssystem soll in ein bestehendes System aus Batterie, Batteriemanagement, Motorcontrollern und Deck integriert und an die Einsatzbedingungen angepasst werden.
		Hieraus ergeben sich bei der Planung zu berücksichtigende konstruktive Rahmenbedingungen.
		%
		\subsection{Bestehendes System}
			%
			Die Batterie besteht aus 40~Lithium-Ionen Zellen in 10S4P-Konfiguration vom Typ \textsc{Samsung INR18650-30Q} mit einer nominellen Zellspannung von \qty{3,6}{\volt}, einer Mindestzellentladekapazität von \qty{2950}{\milli\amperehour} und einem maximalen Entladestrom von \qty{15}{\ampere} (vgl.~\cref{tab:cellspecifications}~\cite{INR18650.30Q.Specs.202202}).
			%
			\begin{table}[h]
				\caption[Zellspezifikationen \textsc{Samsung INR18650-30Q}]{Zellspezifikationen \textsc{Samsung INR18650-30Q}~\cite{INR18650.30Q.Specs.202202}.}%
				\label{tab:cellspecifications}
				\centering
				\begin{tabular}{p{.4\textwidth}l}
					\toprule
					Charakteristik				& Spezifikationen \\ \midrule
					Minimale Entladekapazität	& \qty{2950}{\milli\amperehour} \\
					Nominelle Zellspannung		& \qty{3,6}{\volt} \\
					Standard Ladebedingungen	& CC/CV, \qty{1,5}{\ampere}, \qty{4,2 +- 0,05}{\volt} \\
					Maximale Ladebedingungen	& CC/CV, \qty{4}{\ampere}, \qty{4,2 +- 0,05}{\volt} \\
					Maximaler Dauerentladestrom & \qty{15}{\ampere} bei \qty{25}{\degreeCelsius} \\
					Minimale Zellspannung		& \qty{2,5}{\volt} \\
					Gewicht						& \qty{48}{\gram} \\
					Betriebstemperatur			& Laden: \qtyrange{0}{50}{\degreeCelsius}, Entladen: \qtyrange{-20}{75}{\degreeCelsius} \\ \bottomrule
				\end{tabular}
			\end{table}
			%
			%
			10S4P meint hier jeweils vier Zellen parallel geschaltet mit 10 jener Sub-Zellen elektrisch in Serie (vgl.~hierzu~\cref{fig:battery pack}).
			So ergibt sich eine nominelle Spannung der Batterie von \qty{36}{\volt}, eine Kapazität von \qty{11,8}{\amperehour} und eine gespeicherte Energie von \(\sim \SI{425}{\watthour}\) bei einem maximalen Dauerentladestrom von \SI{60}{\ampere}.
			\begin{figure}[h]
				\centering
				\includegraphics[width=.9\textwidth]{Footage/Pictures/Battery pack.jpg}
				\caption[Der verbaute Lithium-Ionen Zellverband]{Der verbaute Lithium-Ionen Zellverband. Zu erkennen sind die 10 Sub-Zellen bestehend aus jeweils vier Einzelzellen.}\label{fig:battery pack}
			\end{figure}

			%
			Das Drehmoment soll durch Elektromotoren erzeugt werden, die von zwei elektronischen Speed Controllern (ESC)\nomenclature[A]{ESC}{Elektronischer Speed Controller} des Typs \textsc{FSESC 4.12}\footnote{\hspace{1mm} Die wiederum industriell gefertigte 1:1 Nachbauten des populären Open-Source \textsc{VESC} von Benjamin~Vedder sind.} angesteuert werden.
			ESC sind elektronische Komponenten vornehmlich zur elektronischen Kommutation von bürstenlosen Gleichstrommotoren (Brushless Direct Current, BLDC).\nomenclature[A]{BLDC}{Brushless Direct Current}
			\begin{figure}[h]
				\centering
				\includegraphics[width=.9\textwidth]{Footage/Pictures/Electronics.jpg}
				\caption[Eingesetzte ESC]{(1) die eingesetzten ESC vom Typ \textsc{FSESC 4.12}, (2) ein HC-06 Bluetooth Modul, (3) passives Batteriemanagementsystem zur Regelung des Ladestromes, (4) \qty{5}{\volt}~Spannungsregler, (5) elektronischer Ein-Aus-Schalter mit zwei parallel geschalteten \qty{80}{\ampere} Sicherungen, (6) Arduino Nano mit NRF24 Transceiver, (7) Steckverbinder der Hall-Effekt-Sensoren, (8) Phasenanschlüsse der Motoren.}\label{fig:electronics}
			\end{figure}
			Als solche schränken sie die Auswahl der Motortypen zwar nicht exklusiv auf BLDCs ein --~Gleichstrommotoren mit Schleifkontakt sind auch denkbar -- allerdings bieten sie in Kombination mit BLDCs einen deutlich höheren Funktionsumfang.
			Darüber hinaus sind BLDC gegenüber Gleichstrommotoren mit Schleifkontakten effizienter, bieten eine höhere Leistungsdichte, sind bauartbedingt unempfindlich gegenüber Nässe und quasi-wartungsfrei\footnote{\hspace{1mm} Je nach Art der Lagerung. Dies betrifft jedoch ausschließlich die mechanischen Komponenten der Motoren.}.
			%
		\subsection{Einsatzumgebung und -bedingungen}
			%
			Das Gewicht des Fahrers wird inklusive Kleidung und transportiertem Gepäck mit \qty{90}{\kilo\gram} angenommen.
			Zuzüglich \(\sim \qty{5}{\kilo\gram}\) durch Deck, Batterie und Elektronik und pessimistisch geschätzte weitere \qty{5}{\kilo\gram} durch die beiden Trucks zusammen mit dem Antriebssystem ergibt sich ein geschätztes, vom Antriebssystem zu beschleunigendes Gesamtgewicht von \(\sim \qty{100}{\kilo\gram}\).

			Weiter soll die fertige Maschine auf in urbanen Gebieten üblichen Untergründen betrieben werden können.
			Es wird also mit leichten bis moderaten Steigungen und in Form des Bodenbelages mit Asphalt und Kopfsteinpflaster gerechnet.\par\medskip
			%
			Mit Abwesenheit einer Lenkstange und einer ``bauartbedingten'' Höchstgeschwindigkeit von ``nicht weniger als \qty{6}{\kilo\metre\per\hour}'' ist die Maschine nach geltender Verordnung zulassungspflichtig, jedoch nicht zulassungsfähig~\cite{Bundesgesetzblatt.2019}.
			Damit soll die Maschine vorrangig als \textit{Sportgerät} für milde bis sonnige Wetterlagen geeignet sein.
			Extrembedingungen wie Starkregen, Schnee oder Eisglätte finden hier keine weitere Beachtung.

	\section{Designziele}
		Mit in \cref{sec:constructive limitations} genannten Einschränkungen können einige Soll-Forderungen formuliert werden.
		So muss das System\ldots
		\begin{itemize}
			\item \ldots in der Lage sein, mindestens das angenommene Gesamtgewicht von \qty{100}{\kilo\gram} moderate Steigungen hinauf befördern zu können.
			Als Designrichtlinie wird hier eine Steigung von \qty{5}{\percent} festgelegt.
			\item \ldots entlang ebenen Asphaltes auf mindestens \qty{25}{\kilo\metre\per\hour} beschleunigen können\footnote{\hspace{1mm}Womit die Maschine nach derzeitiger deutscher Gesetzeslage nicht mehr als Elektro\textbf{kleinst}fahrzeug kategorisiert werden kann~\cite{Bundesgesetzblatt.2019}.}.
			In Kombination mit obiger Forderung wird hier auf das Festlegen eines Zeitintervalls, innerhalb dessen die Endgeschwindigkeit erreicht werden soll, verzichtet.
			\item \ldots einfach zu Warten sein.
		\end{itemize}
		Neben den harten Zielen ist wünschenswert, dass das System\ldots
		\begin{itemize}
			\item \ldots möglichst aus selbst herstellbaren Komponenten besteht.
			\item \ldots kostengünstig ist.
			Als Richtwert soll hier \dEUR{300} angelegt werden.
			\item \ldots die von der Batterie zur Verfügung gestellte Energie von \SI{425}{\watthour} bezogen auf die erreichbare Reichweite möglichst effizient nutzt.
		\end{itemize}
% LTeX: language=de-DE
\chapter{Einleitung}
	%
	Weltweit findet derzeit auf politischer wie gesellschaftlicher Ebene ein Umdenken im Transportwesen statt --~sei es der Transport von Gütern, Fahrgästen oder im Individualverkehr.
	Angefeuert durch den unmittelbaren monetären Druck durch steigende Treibstoffpreise, die immer deutlicher werdenden Folgen fortdauernder CO\textsubscript{2}-Emissionen aber auch Lebensqualität beeinflussende Faktoren wie Staus und abnehmende Luftqualität in urbanen Gebieten treibt eine wachsende Zahl Menschen aus Auto heraus auf alternative Transportmöglichkeiten.
	Neben der klassischen Möglichkeit des Fahrrades kam mit erheblichen Verbesserungen und der deutlich breiteren Verfügbarkeit der Lithiumbatterie-Technologie ein Wandel des gesellschaftlichen Lebens einher, wie es vergleichbar zuletzt geschah, als das Smartphone die Bühne der Welt betrat -- die Elektrifizierung des Individualverkehrs.
	Nachdem einige Vorreiterstädte bereits früh mit baulichen Maßnahmen etwa durch Herabsetzen innerörtlicher Geschwindigkeitsbegrenzungen, Ausbau von Fahrradwegen oder Zuwachs öffentlicher Verkehrsmittel reagierten, ziehen nun immer mehr Städte nach.
	Dieser wechselseitige Trend bildet sich auch in der politischen Stimmung ab mit einer der wichtigsten Novellen für das öffentliche Verkehrsbild, die die ``\textit{Verordnung über die Teilnahme von Elektrokleinstfahrzeugen am Straßenverkehr und zur Änderung weiterer straßenverkehrsrechtlicher Vorschriften}'' von 2019 mit sich zog~\cite{Bundesgesetzblatt.2019}.
	Sie leutete den Advent breit verfügbarer persönlicher elektrischer Kleinstvehikel (PEKV) zur Überbrückung der ``letzten Meile'' ein.\par\medskip
	%
	Die Kategorie elektrifizierter persönlicher Fahrzeuge lässt sich grob unterteilen in elektrische Personenkraftwagen (ePKW), persönliche elektrische Vehikel (PEV) und --~wie oben bereits erwähnt~-- die persönlichen elektrischen Kleinstvehikel in der kleinsten Variante.
	Zwar dominiert die erste Gruppe gegenwärtige politische Bemühungen zum Thema, gerade im städtischen Raum bieten sie jedoch kaum bis kein Potenzial, Infarkte des Straßenverkehrs zu vermeiden.
	Die Ladeinfrastruktur ist noch nicht einheitlich und flächendeckend geregelt und allem voran sind sie preislich für breite Teile der Bevölkerung unattraktiv.
	Vielversprechender sind Vertreter der beiden letztgenannten Gruppen.
	Dem Statistischen Bundesamt zufolge besaßen zu Jahresanfang~2020 etwa jeder neunte deutsche Haushalt oder \SI{11,4}{\percent} zumindest ein elektrisch angetriebenes Fahrrad.
	Während sie Anfang 2015 mit \(\sim \SI{4}{\percent}\) noch in etwa jedem 25. Haushalt aufzufinden waren kann hier eine Verbreitung um fast das Dreifache verzeichnet werden~\cite{zahl.der.ebikes.StatistischesBundesamt.2020.09.28}.\par\medskip
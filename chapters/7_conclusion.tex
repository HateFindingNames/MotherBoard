% LTeX: language=de-DE
\chapter{Zusammenfassung}
	Die geforderte Maximalgeschwindigkeit konnte zwar im Leerlauf deutlich übertroffen werden, im Rahmen von Feldtests unterlag sie jedoch mit einem Spitzenwert von \qty{23,7}{\kilo\metre\per\hour} leicht den geforderten \qty{25}{\kilo\metre\per\hour}.
	Da sie jedoch softwareseitig limitiert wurde und bei Erreichen der gemessenen Maximalgeschwindigkeit noch ausreichend Leistungsreserven zur Verfügung standen ist davon auszugehen, dass, mit geeignetem Personenschutz höhere Geschwindigkeiten ohne weiteres erzielt werden können.

	Es ist zwar aus dem Stillstand heraus nicht möglich einen Hang hinauf zu beschleunigen, jedoch zeigte sich im Feld, dass bereits eine geringe Anfangsgeschwindigkeit etwa durch leichtes Anschieben ein Anfahren ermöglicht.
	So konnten Steigungen von bis zu \qty{7,5}{\percent} noch gut befahren werden, womit die Anforderungen an das Drehmoment übertroffen wurden.\par\medskip
	%
	Die Materialauswahl der mechanischen Komponenten kann für den Einsatzzweck als gut bezeichnet werden.
	Der bauartbedingte Spalt zwischen Motorflansch und Rotor lässt Eindringen von feinem Staub und Schmutz in das Innere der Motoren und damit an die Wicklungen und in die Spalte zwischen Magnete und Anker zu.
	Um ein frühzeitiges Versagen der Motoren zu verhindern, ist für zukünftige Revisionen ein Schutz in Form eines Schildes oder Ähnlichem zu empfehlen.
	Da der Riemen und die gedruckten Zahnriemenscheiben keine Spuren von lokalen Schäden aufweisen erscheint ein Riemenschutz zwar nicht zwingend erforderlich, mithin lagert sich um die Motorachse allerdings eine nicht unerhebliche Menge Schmutz an.
	Hier bieten sich die im vorgestellten Design bereits vorgesehenen Bohrungen für eine Abdeckung gut an.\par\medskip
	%
	Mit einem durchaus unhandlichen Gewicht des Gesamtsystems von \(\approx \qty{17,5}{\kilo\gram}\) ist anzuraten den Ladezustand im Blick zu halten.
	Händischer Transport über größere Strecken verbietet sich de facto und die hohe Masse erschwert manuelles Beschleunigen deutlich.
	Das Antriebssystem bietet lediglich bei den verwendeten Motoren --~auf Kosten der Leistung --~Einsparpotential bezüglich des Gewichtes.
	%
	Abschließend ist zu sagen, dass das Projekt trotz mangelnder Montagefreundlichkeit ein Erfolg war und --~es wird noch aktiv genutzt --~ist.
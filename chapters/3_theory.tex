% LTeX: language=de-DE
\chapter{Theorie}
	Die mechanische Gesamtleistung, die vom System auf den Boden übertragen werden, muss ist die Summe unterschiedlicher Einzelfaktoren.
	Neben der erforderlichen Leistung, um die träge Masse von Maschine und Pilot aus dem Stand auf eine gewünschte Geschwindigkeit zu beschleunigen, müssen zusätzliche Reserven zur Verfügung stehen, um mechanische Verluste wie Rollwiderstand zum Untergrund, bei höheren Geschwindigkeiten zunehmend aerodynamische Effekte oder Hangabtriebskräfte während des Befahrens von Steigungen überwinden zu können.\par\medskip
	%
	Die Hangabtriebskraft mit dem Neigungswinkel \(\theta\)\nomenclature[G]{\(\theta\)}{Hangneigungswinkel\nomunit{rad}} ist gegeben durch:
	\begin{align}
		F_\text{Hang} = m \cdot g \cdot \sin\left(\theta\right)
		\label{eq:downhill force}
	\end{align}
	\nomenclature[L]{\(F_\text{Hang}\)}{Hangabtriebskraft\nomunit{\newton}}%
	Mit der Umrechnung der im Straßenverkehr üblichen Angaben in \unit{\percent} zu \unit{\radian} durch \(\arctan\left(\frac{\angle}{\qty{100}{\percent}}\right)\)\nomenclature[S]{\(\angle\)}{Hangneigung\nomunit{\percent}} wird obige Gleichung zu
	\begin{equation}
		F_\text{Hang} = m \cdot g \cdot \sin\left(\arctan\left(\frac{\angle}{\qty{100}{\percent}}\right)\right)
		\label{eq:downhill force incline to radian}
	\end{equation}
	\begin{figure}[h]
		\centering
		\includesvg[width=.6\textwidth, inkscapelatex=true]{Calc/torque_incline}
		\caption[Skizze aller wirkenden Kräfte bei einer Fahrt Hangaufwärts]{Skizze aller wirkenden Kräfte bei einer Fahrt Hangaufwärts.}%
		\label{fig:sketch torque incline}
	\end{figure}

	Der Rollwiderstand wird beschrieben durch:
	\begin{align}
		F_\text{Roll} = m \cdot g \cdot c_\text{Roll}
		\label{eq:rolling resistance}
	\end{align}
	\nomenclature[L]{\(F_\text{Roll}\)}{Rollwiderstand\nomunit{\newton}}%
	\nomenclature[L]{\(m\)}{Masse\nomunit{\kilo\gram}}%
	\nomenclature[L]{\(g\)}{Erdbeschleunigung\nomunit{\metre\per\square\second}}%
	\nomenclature[L]{\(c_\text{Roll}\)}{Rollwiderstandkoeffizient\nomunit{1}}%
	mit dem dimensionslosen Rollwiderstandskoeffizienten \(c_\text{Roll}\) der wiederum das Verhältnis aus Rollreibungskoeffizienten \(\mu_\text{Roll}\)\nomenclature[G]{\(\mu_\text{Roll}\)}{Rollreibungskoeffizient\nomunit{\metre}} und dem Radius der Rollen nach \(\frac{\mu_\text{Roll}}{r}\) beschreibt.\par\medskip
	%
	Die durch Reibung bei Durchgang eines Körpers durch ein fluides Medium (hier Luft) verursachte, der Bewegung entgegen gerichtete Kraft errechnet sich aus:
	\begin{align}
		F_\text{Ström} = \frac{1}{2} \cdot c_\text{Ström} \cdot \rho \cdot A \cdot v^2
		\label{eq:air drag}
	\end{align}
	\nomenclature[L]{\(F_\text{Ström}\)}{Strömungswiderstand\nomunit{\newton}}%
	\nomenclature[G]{\(\rho\)}{Gasdichte\nomunit{\kilo\gram\per\cubic\metre}}%
	\nomenclature[L]{\(A\)}{Fläche\nomunit{\square\metre}}%
	\nomenclature[L]{\(c_\text{Luft}\)}{Strömungswiderstandkoeffizient\nomunit{1}}%
	\nomenclature[L]{\(v\)}{Laterale Geschwindigkeit\nomunit{\metre\per\second}}%
	%

	Nun lässt sich mit bekanntem Radius der Rolle und unter Berücksichtigung von \crefrange{eq:downhill force incline to radian}{eq:air drag} für das rückwirkende Drehmoment schreiben:
	\begin{align}
		T_\text{Hang}	&= \left(F_\text{Hang} + F_\text{Roll} + F_\text{Ström}\right) \cdot r \nonumber \\
						&= \left[ m \cdot g \left( \sin\left(\arctan\left(\frac{\angle}{\qty{100}{\percent}}\right)\right) + c_\text{Roll} \right) + \frac{1}{2} \cdot \rho \cdot A \cdot c_\text{Ström} \cdot v^2 \right] \cdot r%
		\label{eq:incline plus roll plus drag torque}
	\end{align}

	\section{BLDC}
		Die Drehmomentkonstante \(K_\text{T}\)\nomenclature[L]{\(K_\text{T}\)}{Drehmomentkonstante\nomunit{\newtonmetre\per\ampere}} eines BLDC berechnet sich aus der reziproken Drehzahlkonstante \(K_\text{V}\)\nomenclature[L]{\(K_\text{V}\)}{Drehzahlkonstante\nomunit{\per\minute\per\volt}} korrigiert um den Umrechnungsfaktor \(\frac{60}{2\pi}\) und trägt die Einheit \unit{\newtonmetre\per\ampere}.
		Der Tatsache Rechnung getragen, dass die Drehzahlkonstante üblicherweise in \unit{\per\minute} angegeben wird, überführt der Faktor \(\frac{60}{2\pi}\) in \unit{\radian\per\second}.
		Der Faktor \(\frac{3}{2}\) entspricht \(\left| 3\cos(\qty{120}{\degree}) \right|\)  \cites{DalY.Ohm.2000}{APPNOTE.yedamale2003brushless}.
		\begin{align}
			K_\text{T}	&= \frac{60}{2\pi} \cdot \frac{3}{2} \cdot \frac{1}{\sqrt{3}} \cdot \frac{1}{K_\text{V}} \nonumber \\
				&\approx 8,27 \cdot \frac{1}{K_\text{V}}
			\label{eq:kv to kt}
		\end{align}
		Die mechanische Untersetzung sei:
		\begin{align}
			\zeta = \frac{N_\text{Rolle}}{N_\text{Motor}}
			\label{eq:reduction}
		\end{align}
		\nomenclature[G]{\(\zeta\)}{Untersetzungsverhältnis\nomunit{1}}%
		\nomenclature[L]{\(N_\text{Rolle}\)}{Zähneanzahl getriebeseitig\nomunit{1}}%
		\nomenclature[L]{\(N_\text{Motor}\)}{Zähneanzahl antriebseitig\nomunit{1}}%
		Das vom System erzeugte, verlustfreie Drehmoment errechnet sich mit:
		\begin{align}
			T	&= K_\text{T} \cdot I_\text{Motor} \cdot \zeta
			\label{eq:frictionless torque}
		\end{align}
		\nomenclature[L]{\(T\)}{Drehmoment\nomunit{\newtonmetre}}%
		\nomenclature[L]{\(I_\text{Motor}\)}{Phasenstrom\nomunit{\ampere}}%
		Die theoretische Maximaldrehzahl und, mit bekanntem Umfang der Rollen, Maximalgeschwindigkeit unter Vernachlässigung von elektrischen und thermischen Verlusten ergeben sich zu:
		\begin{align}
			\omega_\text{max} = K_\text{V} \cdot U_\text{Bat}
			\label{eq:max rpm}
		\end{align}
		\nomenclature[G]{\(\omega_\text{max}\)}{Maximaldrehzahl\nomunit{\per\minute}}%
		und
		\begin{align}
			v_\text{max} = K_\text{V} \cdot U_\text{Bat} \cdot 2\pi r_\text{Rolle} \cdot \frac{0,06}{\zeta}
			\label{eq:max speed km h}
		\end{align}
		\nomenclature[L]{\(v_\text{max}\)}{Maximalgeschwindigkeit\nomunit{\kilo\metre\per\hour}}
		\nomenclature[L]{\(U_\text{Bat}\)}{Batteriespannung\nomunit{\volt}}%
		\nomenclature[L]{\(r_\text{Rolle}\)}{Radius der Rollen\nomunit{\metre}}%
		%
		\nocite{Meschede.2015}\nocite{Demtroder.2018}
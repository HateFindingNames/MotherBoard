% LTeX: language=de-DE
\chapter{Mechanik}
	Wie in \cref{sec:constructive limitations} besprochen beschränkt sich die Auswahl möglicher Antriebselemente praktischerweise auf BLDC-Motoren die wiederum auf dem Markt als \textit{Inrunner} und \textit{Outrunner} erhältlich sind.
	In ersteren sind die Statorwicklungen an der Außenseite angeordnet, zweitere ordnen sie an der Innenseite an.
	\begin{figure}[h]
		\centering
		\includesvg[width=.8\textwidth]{Assets/Inrunner_Outrunner}
		\caption[Gegenüberstellung von Inrunner und Outrunner]{Schematische Gegenüberstellung von BLDC-Motoren als Inrunner (links) bzw. Outrunner (rechts) ausgeführt~\cite{inrunner.outrunner.2022}.}
		\label{fig:inrunner outrunner}
	\end{figure}
	Das Funktionsprinzip bleibt so zwar unverändert, durch den vergrößerten Radius des Angriffspunktes der magnetischen Kopplung kann bei gleicher Baugröße und gleichem Phasenstrom allerdings ein höheres Drehmoment erzeugt werden.
	Sind größere Drehzahlen gefordert, so ist die Bauweise des Inrunners durch den reduzierten Durchmesser des Rotors vorteilhaft.
	Da hohe Drehzahlen gegenüber einem zu erzeugenden Drehmoment für das zu konstruierende Antriebssystem von untergeordneter Priorität sind und sich oberhalb eines Schwellwertes sogar kontraproduktiv auswirken können fällt hier die Wahl auf das Outrunnerprinzip.\par\medskip
	%
	Auch während Lenkmanövern muss eine Übertragung des Drehmomentes auf die Rollen der Trucks sichergestellt sein.
	Praktikabel und mit einfachen Mitteln denkbar sind hier eine koaxiale Positionierung des Motors zur angetrieben Rolle.
	Der Motor kann hier unter eines Offsets weiter im Zentrum des Hangers platziert werden, setzt in dem Fall jedoch eine Hohlwelle zwingend voraus.\par
	Eine weitere Option bildet die Positionierung des Motors in der Rolle selbst.
	Beschichtet mit einem geeigneten Material kann der Rotor so unmittelbar den Kontakt zum Boden herstellen.
	Vorteile dieser Variante sind sowohl eine gute Marktverfügbarkeit, als auch eine deutliche Reduktion mechanischer Komplexität des Antriebssystems.
	Nachteilig sind hier im Vergleich deutlich höhere Preise, weniger Auswahl und schlechte Wartungsmöglichkeiten.\par
	Letztlich kann die Motorachse parallel zur Rollenachse positioniert werden.
	So ist zwar die Kraftübertragung von Welle zu Rolle aufwändiger herzustellen, es werden aber die geringsten Anforderungen an die Motoren bezüglich ihrer Bauform gestellt wodurch sie besonders günstig und in großer Vielfalt am Markt verfügbar sind.\par
	
	%
	\section{Motorbefestigung}\label{sec:motorbefestigung}
		Mit obigen Überlegungen fällt die Lösungswahl auf parallel zur Rollachse angeordnete Outrunnermotoren.
		\begin{figure}[h]
			\centering
			\includesvg[width=.8\textwidth, inkscapelatex=false]{Footage/AwesomeBoard Transmission CAD/Drawings/Mount - Hanger Clamp}
			\caption{The Hanger Clamp}
			\label{fig:hanger clamp drawing}
		\end{figure}
		\begin{figure}[h]
			\centering
			\includesvg[width=.8\textwidth, inkscapelatex=false]{Footage/AwesomeBoard Transmission CAD/Drawings/Mount - Motor Piece - v1}
			\caption{The Motor Piece}
			\label{fig:motor piece drawing}
		\end{figure}
		%
	\section{Transmission}\label{sec:transmission}
		\begin{figure}[h]
			\centering
			\includesvg[width=.8\textwidth, inkscapelatex=false]{Footage/AwesomeBoard Transmission CAD/Drawings/HTD Parametric Pulley}
			\caption{HTD-5M Zahnscheibe}
			\label{fig:htd 5m driven}
		\end{figure}